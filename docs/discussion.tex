\chapter{Discussion}

\section{Participatory GIS}

The land owners who participated in the discussion recommended the use of the Deer Management Plan map to represent each community. However, it was decided that the community council boundaries are more appropriate for this project, as these boundaries are available under an open-source licence in a geospatial data format, which does not require manual digitisation or data requests. Additionally, community councils are statutory bodies, and are therefore more likely to be used in the public sector for spatial analysis.

Through this session, cultural heritage sites were identified as important sites in the study area. The visual impacts on these sites are important as they are present in the landscape and are frequented by locals and visitors alike. Therefore, choosing designated Scheduled Monuments in each community council area as viewpoints was suitable.

For this session, only five land owners attended. To gather public opinion from a wider range of community members affected by development at N4, there is the opportunity to use a web map with editable data layers for large-scale anonymous data collection, which is a form of crowdsourcing. To reach a wider audience, translation of the map elements into Scottish Gaelic will be necessary.

\section{Visual impact and site suitability assessment}

Zonal statistics are used, as the overall cumulative rasters generated for each scenario will have no noticeable differences when viewed at a small map scale. This is due to the size of the land area covered within the study area boundary. Additionally, the relatively large-scale of offshore wind farms compared to other marine structures, such as aquaculture sites, means that even the smallest offshore wind turbine will have a significant impact on the landscape and seascape. Zonal statistics provide the viewshed at each viewpoint, which allows for the scenarios to be compared more effectively.

Based on \autoref{fig:zs_scenarios} and \autoref{fig:zs_elevation}, no clear relationship is found between the elevation of the viewpoint sites and the visual impact. There could be other factors influencing the visibility of the turbines at these sites, such as the aspect, which is worth exploring in the future. Additionally, there is opportunity to repeat the analysis using a \gls{dtm} generated using higher resolution data, or LiDAR elevation models, which will allow comparison of different interpolation methods and data quality on the results.

The scenarios investigated all occupy the entire space of N4 as equidistant points. By combining the site suitability assessment through \gls{mca}, further scenarios which concentrate the turbine installations in the northernmost vertex of N4 could be investigated. This will allow the evaluation of developing a small part of N4 which is deemed more suitable than the rest of N4's area.

This approach is also technologically-neutral, where turbine specifications are not defined in detail, apart from the height. Using turbine manufacturing trends and technology learning curves has the potential to develop many additional scenarios for comparison.
