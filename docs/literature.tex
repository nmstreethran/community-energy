\chapter{Literature review}

\section{Visual impact assessment}

According to NatureScot's guidance on visual representation of wind farms, offshore wind farms are always expected to have a higher impact due to their relatively large sizes compared to onshore turbines \autocite{naturescot-visual}. While a height of more than 150 m is considered large for an onshore wind turbine, offshore wind turbines, based on current technological trends, always exceed this height. As a result, it is very important to quantify the visual impacts of a new offshore wind farm development, and what changes this could cause to existing landscapes and seascapes.

There are a number of mediums that can be used to assess visual impacts, which are geared towards different target audiences. Types of visualisations include wireframes, photomontages, and two-dimensional or three-dimensional maps. In a public participation study conducted by \cite{berry2011}, it was deduced that the public generally prefers realistic visualisations, such as photomontages (due to their true colour and photographic format), as opposed to more abstract representations, such as maps or three-dimensional models. However, it is worth noting that each individual would have a different sensitivity to visual impacts \autocite{falconer2013,appleton2003}, so having different mediums of visualisations to complement each other are important. Additionally, although photomontages are seen as realistic, they tend to be inaccurate as well as subjective \autocite{falconer2013}, and also provide less information than a map with the appropriate symbologies.

When assessing visual impacts, visual receptors as well as the structure that causes the visual impact, must be identified. \cite{falconer2013} conducted a study assessing visual, seascape, and landscape impacts of coastal aquaculture development, using the Outer Hebrides as a case study. The visual receptors identified in this study include tourists and local community members.

A \gls{dtm} forms the basis of constructing these various visualisations. Based on \cite{naturescot-visual}, the same \gls{dtm} is used to generate \gls{ztv} of wind farms, from which other visuals, such as photomontages and wireframes, are constructed. Viewshed analysis, which is defined as a widely-used objective approach to quantifying visual impacts, uses a raster \gls{dtm} to calculate how visible each pixel is on its surface from a viewpoint (\cite{falconer2013}, citing Eastman 2012).

Viewpoint selection is also an important aspect of visual impact assessment. \cite{falconer2013} considered various viewpoints, including buildings, roads, ferry routes, and points of interest in the Outer Hebrides to assess the visual impacts of land and sea aquaculture cages. Such level of detail would not be necessary for an offshore wind farm's visual impact assessment, as wind turbines are magnitudes larger than aquaculture structures both onshore and offshore.

\section{Site suitability assessment}

The basis of \gls{mca}, which may also be referred to as multi-criteria decision analysis (MCDA) or multi-criteria evaluation (MCE), is to consider multiple criteria, which are alternative choices, when making a decision \autocite{dean2020,eastman2005-mca}. In the context of spatial planning, \cite{eastman2005-mca} defines \gls{mca} decisions as either being resource allocation or policy decision problems. Based on this definition, the direct allocation of Scottish waters to construct an offshore wind farm is part of the former, while the Scottish Government's wind energy commitments and the \gls{smp} Plan Options that influence the offshore wind development are part of the latter group. Both examples are the result of considering multiple criteria or alternative decisions.

\gls{mca} is widely-used in \gls{gis} for industrial site selection, and there have been many recent publications (in the last decade) that concern offshore wind site suitability assessment using \gls{mca} \autocite{gaveriaux2019,mekonnen2015,vasileiou2017,tercan2020,deveci2020,mahdy2018,basset2021}. Many of these publications consider factors such as meteorology, technology, electricity grid networks, and geology in determining the suitability of an offshore wind site. Since the focus of this project is community impacts, these factors are omitted from the analysis. The analysis will also not quantify benefits such as clean energy generation, employment, and profits as a result of the development.

The criteria relevant to this project's impact assessment are socio-economic and environmental impacts. \cite{gaveriaux2019} broadly used marine data for their socio-economic criteria, including anchorage, restricted areas, fishing areas, and recreation zones. The former two are classed as constraints, as they physically prevent any development at the area of intersection. The latter two are classed as factors, where weights can be assigned to indicate their relative importance, but they do not necessarily prevent development on their coverage area. Similarly, for their environmental criteria, protected areas and marine reserves are classed as constraints, while commercial fishing resources, fauna, and flora are classified as factors.

\cite{basset2021} categorised the criteria into groups and subgroups of the same theme. For example, under the environmental group, three subgroups were present, which quantified the offshore wind farm's potential to cause damage to the marine ecosystem, water disturbances, and risk of collision to birds. However, the authors do not consider the socio-economic criteria as defined in the previous paper, such as fishing areas. \cite{deveci2020} and \cite{basset2021} use qualitative inputs for determining the acceptance of local communities in their analysis.

A summary of the factors found in these sources can be found in \autoref{tab:lit-criteria}.

\begin{longtable}{lllr}
  \caption[Summary of criteria used for offshore wind site suitability analysis in recently published literature.]{Summary of criteria used for offshore wind site suitability analysis in recently published literature \autocite{gaveriaux2019,mekonnen2015,vasileiou2017,tercan2020,deveci2020,mahdy2018,basset2021}. \label{tab:lit-criteria}} \\

  \toprule
  \multicolumn{1}{l}{\textbf{Criteria}} &
  \multicolumn{1}{l}{\textbf{Type}} &
  \multicolumn{1}{l}{\textbf{Theme}} &
  \multicolumn{1}{l}{\textbf{Frequency}} \\
  \midrule
  \endfirsthead

  \multicolumn{4}{c}
  {{\textbf{\tablename\ \thetable{}} -- continued from previous page}} \\
  \toprule
  \multicolumn{1}{l}{\textbf{Criteria}} &
  \multicolumn{1}{l}{\textbf{Type}} &
  \multicolumn{1}{l}{\textbf{Theme}} &
  \multicolumn{1}{l}{\textbf{Frequency}} \\
  \midrule
  \endhead

  \midrule
  \multicolumn{4}{r}{{Continued on next page}} \\
  \bottomrule
  \endfoot

  \endlastfoot

  Natural restricted areas & constraint & environmental & 5 \\
  Socio-economic and defence restricted areas & constraints & socio-economic & 5 \\
  Anchorage & constraint & socio-economic & 1 \\
  Shore distance & factor & environmental; socio-economic & 6 \\
  Shipping lines & constraint & socio-economic & 3 \\
  Marine ecosystem & factor & environmental & 4 \\
  Recreation zones & factor & socio-economic & 2 \\
  Fishing areas & factor & socio-economic & 2 \\
  Bird flight and habitat & factor & environmental & 2 \\
  Public acceptance & factor & socio-economic & 2 \\
  Population density & factor & socio-economic & 1 \\

  \bottomrule
\end{longtable}

Based on these publications, there are a number of methods for comparing criteria and assigning weights based on their relative importance. A Boolean mask is used when the outputs are binary; for example, restricted areas are constraints, so any area within the constraint is unsuitable. A pairwise comparison is when the criteria are assigned weights by direct comparison in pairs. Fuzzy memberships take into account correlations and relationships between variables of a criteria. Finally, an abundance rank is used as a simple means to aggregate the various criteria.

\section{Public participation in spatial planning}
