\chapter*{Abstract}

This research project investigated the potential impacts of the Sectoral Marine Plan for Offshore Wind Energy option N4 on island communities. Site N4 is situated adjacent the northern shore of the Isle of Lewis, with the closest edge of the plan's area being less than 5 km from the shoreline. As a result, adverse landscape and seascape changes are expected, which is expected to affect the local communities, who rely on small-scale economic activities, land ownership, and tourism as sources of income. To quantify these impacts, this project used a GIS-based methodology, including a viewshed analysis to assess visual impacts, multi-criteria analysis to assess the site's suitability for a commercial-scale offshore wind farm, and a participatory GIS session with community landowners to gain local knowledge and opinions on the large-scale development proposal. The conclusions of this project demonstrated the importance of considering different development scenarios and anticipated impacts, as well as the usefulness of visually and verbally communicating this knowledge to communities that may be affected by large-scale offshore wind development.

\noindent\textbf{\textit{Keywords:}} \keywords
