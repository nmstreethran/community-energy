\chapter{Conclusion}

The aim of this research project was to investigate the impacts of a commercial-scale offshore wind farm development inshore of the Isle of Lewis of the Outer Hebrides in Scotland on the island's communities. The site of concern is called site N4, which is one of the plan options outlined in the Sectoral Marine Plan for Offshore Wind Energy, published by the Scottish Government in October 2020.

Due to the close proximity of this site to the northern shoreline of Lewis, as well as the island being renowned for cultural heritage, scenic views, and a distinctive coastline, it is expected to cause a number of adverse impacts on the local communities. The communities rely heavily on tourism, land ownership, and crofting as main economic activities, which are in danger of being drastically impacted due to landscape and seascape changes pertaining to the wind farm development.

To assess the magnitude of these impacts, this project utilised open-source data and geospatial software packages to conduct a viewshed analysis for visual impact assessment, a multi-criteria analysis (\gls{mca}) to assess the offshore wind site suitability, and a participatory GIS session to gather opinions and local knowledge from community members regarding development at site N4.

Through the visual impact assessment, it was deduced that taller turbines are anticipated to produce a larger visual impact, while there is no clear correlation between the number of turbines and visual impact. From the \gls{mca}, the northern areas of site N4 were found to be the most suitable for commercial-scale offshore wind farm development, as they are furthest away from populations, the coast, and scenic areas. Developing a small portion of site N4 towards the northern areas may be a compromise between minimising community impacts and increasing renewable energy genration. The participatory GIS session with landowners affected by site N4 showed that the use of GIS tools, such as web maps and geovisualisations, in stakeholder sessions with community members can serve as useful visual aid to help them understand the scale and impacts of large-scale offshore wind farm development, as well as increase their engagement.
