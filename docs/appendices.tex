\chapter{Data \label{app:data}}

\section*{Dataset references}

References for datasets, data catalogues, and \glspl{api} used.

\printbibliography[heading=none,keyword=data]

\section*{Copyright information and disclaimer}

This project uses public sector information licensed under the Open Government
Licence (OGL) v3.0:
\url{https://www.nationalarchives.gov.uk/doc/open-government-licence/version/3/}.

\begin{itemize}[noitemsep]
  \item Contains Ordnance Survey data, Royal Mail, National Statistics,
  Scottish Government, Marine Scotland, and National Records of Scotland data.
  \textcopyright~Crown copyright and database right 2021.
  \item Contains Historic Environment Scotland data. \textcopyright~Historic
  Environment Scotland - Scottish Charity No. SC045925 2021.
  \item Contains Scottish Natural Heritage information.
  \textcopyright~NatureScot 2021.
  \item Contains Scottish local authority data from the Spatial Hub.
  \textcopyright~Improvement Service 2021.
\end{itemize}

Bathymetry data used in this project was made available by the EMODnet
Bathymetry project, funded by the European Commission Directorate General for
Maritime Affairs and Fisheries. The data originators are the United Kingdom
Hydrographic Office (UKHO), OceanWise Limited, and the General Bathymetric
Chart of the Oceans (GEBCO).

All data used in this project are not to be used for navigation or for any
other purpose relating to safety at sea.

\chapter{Software \label{app:soft}}

\section*{Python scripts}

Python scripts used in the analysis are available in the following GitHub
repository:
\url{https://github.com/nmstreethran/community-energy}.

\section*{Software references}

References for software packages and \glspl{api} used in this project.

\printbibliography[heading=none,keyword=software]

\chapter{Participant materials \label{app:pgis}}

The following pages contain the participant information sheet and consent form
for the participatory \gls{gis} session.

The participatory session was in the form of a two-hour meeting with five
landowners held online through Microsoft Teams.

The ArcGIS Online web map used during this session can be accessed using the
following link:
\url{https://www.arcgis.com/home/webmap/viewer.html?webmap=43bce1bc4a9b4faabb250838d6d3ec3f}.

\newpage

{
\renewcommand\familydefault{\sfdefault}
\normalfont
\setlength{\parindent}{0pt}
\singlespacing

\begin{flushright}
\footnotesize
\bfseries
Department of Geography \& Environment \\
School of Geosciences \\
University of Aberdeen
\end{flushright}

\vspace{30pt}
\begin{center}
\textbf{\MakeUppercase{Participant information sheet}}

Mapping the impacts of a proposed offshore wind development plan on Isle of
Lewis communities
\end{center}

\vspace{30pt}
My name is Nithiya Streethran. I am a Master of Science (MSc) student in
Geographical Information Systems (GIS) at the Department of Geography \&
Environment, University of Aberdeen\footnote{\url{https://www.abdn.ac.uk/}}. I
would like to invite you to consider participating in the research project
\textbf{``Mapping the impacts of a proposed offshore wind development plan on
Isle of Lewis communities''}. Below is some information about the project, to
help you decide whether you would like to take part.

\textbf{Participation in the research project is completely voluntary. You can
withdraw from the project at any time, without having to give a reason.}

\vspace{30pt}
\textbf{\MakeUppercase{Aims}}

The Sectoral Marine Plan for Offshore Wind Energy%
\footnote{\url{https://www.gov.scot/publications/sectoral-marine-plan-offshore-wind-energy/}},
published by the Scottish Government in October 2020,
shows a number of commercial-scale offshore wind development plan options
around Scotland. One of these options, called N4, is significantly close to the
northern/northwestern coast of the Isle of Lewis in the Outer Hebrides.
According to the plan, N4 has a total area of 200 km\textsuperscript{2} and can
potentially generate a maximum of 1 gigawatt (GW) of wind power. The closest
edge of the N4 site is less than 5 km away from the shore. The site is in close
proximity to a number of towns with community land ownership, sites of
historical and natural significance, and tourist attractions.

This study aims to evaluate the visual and environmental impacts of a number of
offshore wind farm development scenarios in site N4 on the coastal areas
affected in the Isle of Lewis. A number of \gls{gis} methods will be utilised,
including multi-criteria analysis, viewshed analysis, and participatory
\gls{gis}. The latter involves a session in which you will participate
alongside other affected landowners to provide inputs as a group. The potential
participants are landowners from [redacted].

Through the participatory \gls{gis} session, I aim to obtain your views and
local knowledge on a number of scenarios of the proposed offshore development
and the affected areas, including coastal activities, and scenic views. Your
inputs will be valuable in improving my initial evaluation of the site and
impacts. All of these will form the basis of an interactive and accessible web
map to further improve community engagement, transparency in the decision
making process, and inform all stakeholders should there be further discussion
on the development of site N4.

\textbf{Note:} This research project will be submitted to the University of
Aberdeen as my dissertation in partial fulfilment of the requirements for the
degree of MSc in GIS.

\vspace{20pt}
\textbf{\MakeUppercase{What you will be asked to do}}

I will first distribute this information sheet and the accompanying consent
form to you on Tuesday, 29th June 2021. Once you go through both documents, and
if you are willing to participate in this research project, you may fill in and
sign the consent form. If you have any questions or concerns regarding this
project, you may contact me via email%
\footnote{\href{mailto:n.streethran.20@abdn.ac.uk}%
{n.streethran.20@abdn.ac.uk}\label{fn:email}}. Please return the completed
forms to me prior to the start of the participatory GIS session, if you wish to
participate.

On Wednesday, 30th June 2021 at 14:00, we will have an online participatory GIS
session where you and fellow landowners will be presented with scenarios of the
proposed development and asked to provide your inputs as a group. This session
will be conducted using an online video conferencing platform (Microsoft Teams%
\footnote{\url{https://www.microsoft.com/en-us/microsoft-teams/group-chat-software}}).
Additionally, you will have access to an online map created using ArcGIS Online%
\footnote{\url{https://www.esri.com/en-us/arcgis/products/arcgis-online/overview}},
which will allow you to visualise the development scenarios and
affected areas, as well as add your local knowledge by drawing directly on the
map.

The online meeting and web map can be accessed through a modern web browser
(such as Mozilla Firefox, Microsoft Edge, and Google Chrome). You may
alternatively choose to use the Microsoft Teams application for the online
meeting. You will be sent an invitation with a link to the meeting and web map
via email.

The session is expected to take approximately two hours of your time.

\vspace{20pt}
\textbf{\MakeUppercase{Risks}}

To ensure that the discussion remains relevant to the topic, is within the
scope of the project, and is not affected by differing views and opinions, the
session will be structured with a series of questions, detailed maps, and
visualisations. I will also provide you with instructions on how to access,
view, interpret, and add data to the web map.

\textbf{Note:} You may refrain from providing answers to any particular
question or withdraw from this session at any point. If you have made a mistake
in the web map, please let me know and I will remove the incorrect inputs.

\vspace{20pt}
\textbf{\MakeUppercase{Data management and storage}}

The audio and video of our interactions will be recorded using Microsoft Teams,
which is provided by the University of Aberdeen and is General Data Protection
Regulation (GDPR) compliant. This data will only be stored securely offline on
my computer's hard drive and will not be shared with other parties (other than
the research team). I will use these recordings to produce an anonymised
transcript, which I will share with you and your fellow participants for
verification. You can request changes to this or request your data to be
removed if you wish.

During the participatory session, you will be able to interact with the web map
and add your local knowledge in the form of points, lines, and areas. This data
will be fully anonymous and synchronised with the online map's database. You
may refer to the ArcGIS Platform's GDPR guidance%
\footnote{\url{https://www.esriuk.com/en-gb/legal/gdpr/guidance-on-the-arcgis-platform}},
published by Esri UK \& Ireland, for more information about
how the ArcGIS Online web map's data is protected. Once the session is
complete, I will download the database and remove the data you entered from the
online server.

At the end of the project and subject to your approval, a fully-anonymised
database generated using your inputs may be published on the Aberdeen
University Research Archive\footnote{\url{https://aura.abdn.ac.uk/}} and/or
GitHub\footnote{\url{https://github.com/}} with
Zenodo\footnote{\url{https://zenodo.org/}}, as an open dataset under the terms
of a Creative Commons Attribution 4.0 License%
\footnote{\url{https://creativecommons.org/licenses/by/4.0/}}.

The project is expected to be completed on Friday, 6th August 2021. Any
personal information collected (i.e. email exchanges, participant consent
forms, and the Microsoft Teams meeting recording) will be deleted on this day.

\vspace{20pt}
\textbf{\MakeUppercase{Confidentiality and anonymity}}

The University's Privacy Notice for Research Participants is available on the
University of Aberdeen's website%
\footnote{\url{https://www.abdn.ac.uk/about/privacy/research-participants-938.php}}.

As mentioned above, raw data and the identity of participants will not be
released to anyone outside the research team. The data you provide will be
analysed and may be used in publications, dissertations, reports or
presentations derived from the research project, but this will be done in such
a way that your identity is not disclosed.

\vspace{20pt}
\textbf{\MakeUppercase{Consent}}

If you agree to take part in the research, you will be asked to indicate your
consent by ticking boxes and adding your signature to the consent form
enclosed below.

\vspace{20pt}
\textbf{\MakeUppercase{Sponsors}}

In addition to the University of Aberdeen, this research is also done in
collaboration with Community Energy Scotland (CES)%
\footnote{\url{https://www.communityenergyscotland.org.uk/}}. CES is an
independent registered charity in Scotland which provides support to
communities for green energy development. I have not received any financial
support from the University of Aberdeen, CES, or any other organisation or
funder to conduct this research project.

\vspace{30pt}
Thank you for considering taking part in this research.

If you have any questions about this research please contact me via
email\textsuperscript{\ref{fn:email}}.

For any queries regarding ethical concerns you may contact the Convener of the
Physical Sciences \& Engineering Ethics Board at the University of Aberdeen%
\footnote{\url{https://www.abdn.ac.uk/staffnet/research/ethical-review-10645.php}}.

\vspace{20pt}
This research project was approved by the Physical Sciences \& Engineering
Ethics Board on 28th June 2021.

\newpage

\begin{center}
\Large
\textbf{Mapping the impacts of a proposed offshore wind development plan on
Isle of Lewis communities}
\end{center}

\vspace{24pt}
\textbf{Consent form for participation in the research project ``Mapping the
impacts of a proposed offshore wind development plan on Isle of Lewis
communities''}.

\textit{Please complete the form below by ticking the relevant boxes and
signing on the line below. A copy of the completed form will be given to you
for your own record.}

\vspace{24pt}
\textbf{Please Tick Box}

\begin{itemize}[label=\ding{112}]
  \item I confirm that the research project \textbf{``Mapping the impacts of a
  proposed offshore wind development plan on Isle of Lewis communities''} has
  been explained to me. I have had the opportunity to ask questions about the
  project and have had these answered satisfactorily.
  \item I consent to the material I contribute being used to generate insights
  for the research project \textbf{``Mapping the impacts of a proposed offshore
  wind development plan on Isle of Lewis communities''}.
  \item I understand that my participation in this research is voluntary and
  that I may withdraw from the project at any time (until the point of data
  analysis) without providing a reason.
  \item I consent to allow the \underline{fully anonymised} data to be used for
  future publications and other scholarly means of disseminating the findings
  from the research project.
  \item I understand that the information/data acquired (including audio/video
  recordings) will be securely stored by researchers, but that appropriately
  anonymised data may in future be made available to others for research
  purposes. I understand that the University may publish appropriately
  anonymised data in its research repository for verification purposes and to
  make it accessible to researchers and other research users.
  \item I agree to participate and provide my inputs as a group with the other
  landowners taking part in this participatory session.
  \item I agree to take part in the above project entitled \textbf{``Mapping
  the impacts of a proposed offshore wind development plan on Isle of Lewis
  communities''}.
\end{itemize}

\vspace{50pt}
\begin{center}
  \footnotesize
  \rule{100pt}{.5pt} \hfill \rule{100pt}{.5pt} \hfill \rule{100pt}{.5pt}\\
  Name of participant \hfill Date \hfill ~~~~~~~~~~~~~~~~ Signature \\[35pt]
  \rule{100pt}{.5pt} \hfill \rule{100pt}{.5pt} \hfill \rule{100pt}{.5pt}\\
  Name of researcher \hfill Date \hfill ~~~~~~~~~~~~~~~~ Signature
\end{center}
}
